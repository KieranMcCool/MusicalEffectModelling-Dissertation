\documentclass{l4proj}
\usepackage{blindtext}
\usepackage[backend=biber, sorting=nyt, style=apa]{biblatex}
\usepackage{hyperref}
\addbibresource{dissertation.bib}

\providecommand{\tightlist}{%
  \setlength{\itemsep}{0pt}\setlength{\parskip}{0pt}}

\title{A Deep Learning Approach to Musical Effects}
\author{Kieran McCool}

\begin{document}

\maketitle

\tableofcontents
\hypertarget{introduction}{%
\chapter{Introduction}\label{introduction}}

\hypertarget{musical-effects}{%
\section{Musical Effects}\label{musical-effects}}

Musical effects are transformations which can be applied to an audio
track to change the sound in some way. These can take the form of simple
tweaks to the frequency range of the track to applying pitch modulation
and beyond.

Musical effects can completely change the way a track sounds and the art
of mixing a track is - in many ways - more complicated than the
composition and performance of the track. Sound engineers have a huge
range of choices and responsibility when it comes to getting the
soundscape correct for the final product.

The music industry is one of the few domains which is still to fully
embrace digital technology. Many recording studios and musicians still
make use of analogue equipment such as tape, vinyl, and vacuum tubes. As
these technologies become more outdated and niche, the cost of
maintaining and replacing equipment rises. As such, it is essential that
software modelling catches up to the performance and quality of these
older technologies.

\hypertarget{musical-effect-modelling}{%
\section{Musical Effect Modelling}\label{musical-effect-modelling}}

The main standard for producing digital effects for use with audio is
through the Virtual Studio Technology (VST) protocol. This protocol
defines a standardised definition for how a VST host would behave,
allowing VST Plugins to be created which manipulate or use the audio
track in some way.

The technology itself is fairly flexible and allows everything from
audio visualisation tools to CPU intensive transformations to be
applied. These effects can also be stacked and ordered easily and
intuitively.

Most effects however, are limited to digital signal processing (DSP)
based transformations, the likes of which are unable to replicate the
non-linearity of many analogue methods. This is cause for concern in the
music industry and is the reason many professionals and hobbyists claim
that they are inadequate for anything more than practice and
experimentation.

\hypertarget{machine-learning}{%
\section{Machine Learning}\label{machine-learning}}

In recent years, a huge amount of research has gone into machine
learning. Its application frequently makes news for revolutionising the
various industries it is applied to. As such, it would be interesting to
see the effectiveness of machine learning in this field.

Of particular interest to us were convolutional neural networks and
Long-Short Term Memory (LSTM) networks, which both seem to be at the
forefront of deep learning research. However, the application of machine
learning to audio data is in its infancy, resulting in quite a
challenging implementation with many layers to it and several
dependencies.

The idea of the project is to create two audio tracks, one which is a
clean signal from an instrument and one which is the result of applying
a VST effect to that clean track. We would then create a neural network
which would take samples from the clean signal and be trained to produce
the value of the corresponding sample in the processed track.

\hypertarget{background}{%
\chapter{Background}\label{background}}

\hypertarget{nebula-vst}{%
\section{Nebula VST}\label{nebula-vst}}

Nebula yields the best results current modelling technologies can
achieve. It achieves this using Volterra Kernel Modelling to better
represent non-linear behaviours and exhibiting some level of memory
capacity. This allows it to model time-based effects and non-linearity
more effectively than primitive DSP approaches.

In doing this, it has become one of the most praised pieces of modelling
software, criticised only for its system requirements and its steep
learning curve. Requiring a minimum of 8GB of RAM with a recommended
amount of 16-128GB, it is a rather demanding piece of software.

Given that Nebula is the best plugin available at this time, for this
project to be considered a success, it should be as good as or better
than Nebula.

\hypertarget{project-magenta}{%
\section{Project Magenta}\label{project-magenta}}

Project Magenta is a venture by the Google Brain Team, their goal is to
explore the use of machine learning in creating art, with a particular
interest in music. Most of their research is in audio synthesis with
some ventures into genre classification and similar problems.

However, this project does raise some interesting questions about the
ability of deep learning to characterise the complexities of music.

Most of their work however, was limited to interacting with MIDI data
rather than with raw audio samples, perhaps limiting the complexity of
the problem, making it an unfair comparison in terms of complexity to
effect modelling.

\hypertarget{wavenet}{%
\section{WaveNet}\label{wavenet}}

Another use of deep learning for audio synthesis is in the field of
generating realistic text-to-speech. DeepMind's WaveNet is a generative
network, trained using raw audio samples to predict the value of the
next sample in the series.

The project found that while generally an LSTM is better suited to this
kind of time-series problem, they could achieve similar results using
stacked convolutional layers applying dilated casual convolutions. At
each layer they would double the dilation amount up to a maximum before
starting again at 1. This a increases the receptive field of the network
in a similar way to how an LSTM selectively remembers traits while
maintaining the ease of training of a convolutional network.

Another way in which the WaveNet implementation differs from a standard
convolutional network is in its output layer. Instead of predicting a
sample as a floating point value representing the amplitude, the output
is a one-hot encoded vector where the largest index corresponds with the
\(\mu\)--Law encoded integer of the sample.

\begin{center}

    $F(x) = sgn(x)\frac{ln(1+\mu|x|)}{ln(1+\mu)} -1 <= x <= 1$

    $\mu$ Law as expressed in \LARGE{SOURCE THIS}
\end{center}

This network has proven itself to be capable of generating extremely
effective, almost human sounding text-to-speech and is far easier to
train than a standard LSTM-based approach.

\hypertarget{aims}{%
\chapter{Aims}\label{aims}}

\hypertarget{effect-choices}{%
\section{Effect Choices}\label{effect-choices}}

The aim of this project is to explore the viability and success of deep
learning in effect modelling at a fairly general level. As such, a
multitude of different effects should be tested.

While there is a huge range of commonly used effects, and an even larger
range of different variations of these same effects with subtly
different responses, most of them fit in to one of three groups.

\begin{center}
\begin{tabular}{ |c|c|c| } 
 \hline
    \textbf{Amplitude Based} & \textbf{Frequency Based} & \textbf{Time Based} \\ 
 \hline
 Distortion & High/Low Pass Filter & Chorus \\ 
 Fuzz & Pitch Shift/Octave & Delay \\ 
 Compression & & Reverb \\ 
 Amplifier/Cab Simulation & & \\ 
 \hline
\end{tabular}
\end{center}

Narrowing the scope of the project to include only these effects results
in a good indication of viability across a multitude of effects, while
keeping the project scope at an attainable level.

\hypertarget{deep-learning}{%
\section{Deep Learning}\label{deep-learning}}

\begin{itemize}
\tightlist
\item
  Growing use of machine learning across all manner of disciplines

  \begin{itemize}
  \tightlist
  \item
    Image recognition, speech synthesis, text-to-speech.
  \item
    Why not audio signals?
  \end{itemize}
\item
  Project aims to investigate the viability and explore the limitations
  of machine learning as a method for modelling effects.

  \begin{itemize}
  \tightlist
  \item
    Success would involve being able to convincingly reproduce the
    characteristics of a given effect after being trained on a clean
    input and the processed signal.
  \item
    If the network can run over unseen data and produce the effect
    reliably.
  \item
    Different measures of effectiveness, but more on this in evaluation.
    Perhaps introduce here though.
  \end{itemize}
\end{itemize}

\textbf{Sections below are unfinished}

\hypertarget{methods}{%
\chapter{Methods}\label{methods}}

\begin{itemize}
\tightlist
\item
  Network Architecture
\item
  Evaluating outputs, Mel Spectogram etc
\end{itemize}

\hypertarget{implementation}{%
\chapter{Implementation}\label{implementation}}

\begin{itemize}
\tightlist
\item
  PyTorch
\item
  SciPy.signal
\item
  LibRosa
\item
  Reaper and the VSTs used
\item
  How the pieces fit together
\end{itemize}

\hypertarget{results}{%
\chapter{Results}\label{results}}

\begin{itemize}
\tightlist
\item
  Split into quantitative and qualitative
\item
  Signal Analysis (Quantitative)
\item
  Spectogram GIFs (Quantitative)
\item
  Loss over time / Standard Deviation of loss ( ??? - Loss hasn't really
  proven to be particularly meaningful)
\item
  ABX tests with classmates/friends (Qualitative)
\end{itemize}

\hypertarget{conclusion}{%
\chapter{Conclusion}\label{conclusion}}

\hypertarget{references}{%
\chapter{References}\label{references}}
\end{document}
